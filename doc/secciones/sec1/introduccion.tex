\section{Introducción}

Esta es la tercera práctica de la asignatura \textit{Inteligencia de Negocio}, impartida en la Universidad de Granada, en el primer semestre del curso 2025/2026.

En esta práctica se propone participar en una competición de la página Zindi, en la que... 

% -------------------------------------------------------------------------------------------------------------------------------------------------- %

\subsection{Descripción del problema}

El problema tratado se denomina `\href{https://zindi.africa/competitions/zindiweekendz-learning-urban-air-pollution-challenge}{Urban Air Pollution Challenge}'. El objetivo de este es predecir la concentración de materia particulada PM2.5 en el aire cada día para cada ciudad a partir de indicadores meteorológicos y datos de observación satelital.

Para ello, disponemos de un conjunto de datos tabulares, interpretados \href{https://zindi.africa/competitions/zindiweekendz-learning-urban-air-pollution-challenge/data}{aquí}. Estos han sido recogidos durante un periodo de 3 meses en distintas ciudades del globo. Este \textit{dataset} no está completo al 100\%: algunas localizaciones no tienen lecturas para un día en particular, y hay muchos datos vacíos (especialmente en los datos de CH4).
 
% -------------------------------------------------------------------------------------------------------------------------------------------------- %

\subsection{Software y estructura del proyecto}

Trabajaremos con Python, concretamente en cuaderno \textbf{marimo} \cite{Agrawal_marimo_-_an_2023}, que proporciona un entorno interactivo y reproducible para la creación de cuadernos ejecutables. Esta herramienta permite combinar código, texto y visualizaciones de forma declarativa, facilitando tanto la exploración como la documentación del proceso de desarrollo.

Para gestionar las dependencias y el entorno del proyecto se ha usado el gestor de paquetes \textbf{uv} \cite{uv}, que ofrece un sistema de resolución y aislamiento extremadamente rápido. Además, uv permite crear entornos virtuales ligeros, instalar versiones específicas de paquetes y reproducir configuraciones de forma determinista, asegurando así la portabilidad y la consistencia del entorno de ejecución en distintas máquinas.

Se ha creado un cuaderno para experimento: \code{regression01.py}, \code{regression02.py}, etc., en el directorio \code{notebooks/}. Puede ejecutar estos archivos tanto como cuaderno como archivo normal de Python. Se recomienda ejecutar como cuaderno para mayor interactividad: 

\begin{lstlisting}[language=bash, backgroundcolor=\color{backcolour}, basicstyle=\ttfamily\footnotesize, frame=single]
uv run marimo edit notebooks/regression01.py
\end{lstlisting}