\section{Análisis exploratorio de datos}

Primero, se ha realizado un análisis exploratorio de los datos (véase en más detalle en el cuaderno \code{EDA.py}). Algunas consideraciones relevantes son:

\begin{itemize}
    \item El conjunto de entrenamiento tiene 30 557 registros, con identificadores, características y valor objetivo.
    \item El conjunto test tiene 16 136 registros, con identificadores y características.
    \item Hay un total de 74 características, correspondientes a los indicadores meteorológicos.
    \item De las 74 características, 68 tienen valores nulos, tanto en el conjunto entrenamiento como en test. De ellas:
    \begin{itemize}
        \item 35 son nulas en menos de un 7\% de los registros,
        \item 26 son nulas en más del 7\% y menos del 28\% de los registros, y
        \item 7 (las referentes a `CH4') son nulas en más del 80\% de los registros.
    \end{itemize}

    \item El rango de fechas del conjunto de datos (tanto en entrenamiento como en test) va del 2 de enero al 4 de abril, haciendo un periodo total de 94 días (véase en la Figura \ref{fig:dates_distribution}).
    \item Existe aproximadamente el mismo número de registros de cada fecha en el conjunto de entrenamiento. 
    \item Tanto el conjunto de datos de entrenamiento como el de test tienen aproximadamente la misma proporción de ejemplos de cada fecha.
    
\begin{figure}[h!]
    \centering
    \fbox{\includegraphics[width=0.9\textwidth, keepaspectratio]{secciones/sec2/imagenes/dates_distribution.png}}
    \caption{
        Gráfico de líneas temporal con las fechas del conjunto de entrenamiento y test.
    }
    \label{fig:dates_distribution}
\end{figure}

    \item Hay 340 localizaciones distintas en el conjunto de entrenamiento. Cabe comentar:
    \begin{itemize}
        \item Más del 50\% de ellas tienen registros en cada uno de los 94 días.
        \item Más del 95\% de localizaciones tienen al menos 69 registros.
        \item El mínimo número de registros de una localización es 3.
    \end{itemize}
    En el conjunto test se repite este patrón, si bien el número mínimo de registros para una localización es 24. 
    
    \item No se repiten localizaciones del conjunto de entrenamiento en test.
    
    \item Los valores objetivo presentan asimetría positiva (véase la Figura \ref{fig:target_values}): la mayoría de valores se concentran en valores bajos (PM2.5 $<200$), pero existe un pequeño grupo con valores extremadamente altos.
    
\end{itemize}

\begin{figure}[h!]
    \centering
    \fbox{\includegraphics[width=0.9\textwidth, keepaspectratio]{secciones/sec2/imagenes/target_distribution.png}}
    \caption{
        Histograma de valores objetivo del conjunto de entrenamiento.
    }
    \label{fig:target_values}
\end{figure}